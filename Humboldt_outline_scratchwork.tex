\documentclass[11pt,a4paper]{article}
\usepackage{graphicx}
% uncomment according to your operating system:
% ------------------------------------------------
\usepackage[latin1]{inputenc}    %% european characters can be used (Windows, old Linux)
%\usepackage[utf8]{inputenc}     %% european characters can be used (Linux)
%\usepackage[applemac]{inputenc} %% european characters can be used (Mac OS)
% ------------------------------------------------
\usepackage{authblk}
\usepackage[superscript]{cite}
\usepackage[document]{ragged2e}
\usepackage[T1]{fontenc}   %% get hyphenation and accented letters right
\usepackage{mathptmx}      %% use fitting times fonts also in formulas
% do not change these lines:
\pagestyle{empty}                %% no page numbers!
\usepackage[left=35mm, right=35mm, top=15mm, bottom=20mm, noheadfoot]{geometry}
%% please don't change geometry settings!

\usepackage{fullpage}
\usepackage{amsfonts}
\usepackage{graphicx}
\usepackage{float}
\usepackage{amsmath}
\usepackage{chemfig}
\usepackage{indentfirst}
\usepackage{longtable}
\usepackage{array}
\usepackage{cellspace}
\usepackage{palatino}
%\usepackage{breqn}
\usepackage{amssymb}
\usepackage{verbatim}
\usepackage[colorlinks=true,citecolor=blue,linkcolor=blue]{hyperref}
\usepackage{siunitx}
\usepackage{xr}

% italicized boldface for math (e.g. vectors)
\newcommand{\bfv}[1]{{\mbox{\boldmath{$#1$}}}}
% non-italicized boldface for math (e.g. matrices)
\newcommand{\bfm}[1]{{\bf #1}}          

%\newcommand{\bfm}[1]{{\mbox{\boldmath{$#1$}}}}
%\newcommand{\bfm}[1]{{\bf #1}}
\newcommand{\expect}[1]{\left \langle #1 \right \rangle} % <.> for denoting expectations over realizations of an experiment or thermal averages

\newcommand{\var}[1]{{\mathrm var}{(#1)}}
\newcommand{\x}{\bfv{x}}
\newcommand{\y}{\bfv{y}}
\newcommand{\f}{\bfv{f}}

\newcommand{\hatf}{\hat{f}}

\newcommand{\bTheta}{\bfm{\Theta}}
\newcommand{\btheta}{\bfm{\theta}}
\newcommand{\bhatf}{\bfm{\hat{f}}}
\newcommand{\Cov}[1] {\mathrm{cov}\left( #1 \right)}
\newcommand{\T}{\mathrm{T}}                                % T used in matrix transpose

\newcommand\blfootnote[1]{%
	\begingroup
	\renewcommand\thefootnote{}\footnote{#1}%
	\addtocounter{footnote}{-1}%
	\endgroup
}


% begin the document
\begin{document}
	\thispagestyle{empty}
	%make title bold and 14 pt font (Latex default is non-bold, 16 pt)
	\title{\Large \textbf{Humboldt Research Fellowship}}
	\author[1]{\large {\underline{Richard Messerly}}}%%[12 pt regular, presenting speaker underlined]
	
	\affil[1]{\textit{Thermodynamics Research Center (TRC), National Institute of Standards and Technology (NIST),
			Boulder, Colorado, 80305, USA}}
	
	\date{} % <--- leave date empty
	\maketitle\thispagestyle{empty} %% <-- you need this for the first page
	\begin{center}
		\title{\textbf{ABSTRACT}}\centering{}
	\end{center}
	\justify

Jadran, I have included outlines for two different proposals. The first proposal is to use more realistic nonbonded potentials to improve the high pressure extrapolation. The second proposal is to move beyond the Lorentz-Berthelot combining rules to improve the prediction of mixture properties. I would appreciate your insight regarding which of these is more likely to be funded and which is more important and/or interesting.

%\section{Outline: Developing more accurate nonbonded potentials for EOS development}
%
%\begin{enumerate}
%	\item Background to establish need for hybrid data sets
%	\begin{enumerate}
%		\item Design of efficient and reliable chemical processes requires accurate equations of state over a wide range of temperatures and pressures
%		\item The quantity and quality of experimental data is insufficient for most compounds
%		\item Molecular simulation is an ideal approach for overcoming the data deficiency at extreme temperatures and pressures
%		\item Improved EOS are developed by supplementing experimental data with simulation results
%	\end{enumerate}
%	\item Reliability of hybrid data set approach depends on accuracy of force field, specifically, how well the model extrapolates to extreme temperatures and pressures
%	\begin{enumerate}
%		\item Recent work has demonstrated that the united-atom Lennard-Jones $n$-6 force field does not extrapolate from vapor-liquid equilibria conditions to elevated pressures
%		\item Improved extrapolation requires a more physically realistic nonbonded potential
%		\item For this reason, we propose the use of an extended Lennard-Jones (ex-LJ, 12-10-8-6) potential
%	\end{enumerate}
%    \item The ex-LJ 12-10-8-6 potential is more flexible than the LJ 12-6, and more theoretically justified than the Mie $n$-6
%    \item Although the ex-LJ was proposed over two decades previously, it has not been tested as extensively as the LJ 12-6, Mie $n$-6, and exponential-6
%	\item One reason for the lack of popularity is that the higher dimensional parameterization of the ex-LJ is computationally infeasible when implementing traditional iterative force field optimization approaches
%	\begin{enumerate}
%		\item Multistate Bennett Acceptance Ratio (MBAR) combined with basis functions (BF) has been shown to greatly accelerate force field parameterization
%		\item MBAR reweights configurations sampled with a reference force field to predict thermodynamic properites for a non-simulated force field
%		\item Basis functions store the contributions from the $r^{-m}$ terms for the nonbonded energies and forces
%	\end{enumerate}
%	\item The methodology proposed in this study is to:
%	\begin{enumerate}
%		\item Develop an EOS over the temperatures and pressures where reliable experimental data exist
%		\item Fit the ex-LJ parameters to EOS derivative Helmholtz energy properties in range where EOS is reliable
%		\item Perform molecular simulations at state points where experimental data are not available
%		\item Iterate:
%		\begin{enumerate}
%			\item Refit the EOS to the hybrid data set
%			\item Reoptimize the force field parameters to match EOS (using MBAR so that no additional molecular simulations are required)
%		\end{enumerate}
%	\end{enumerate}
%	\item With this iterative approach, it is possible to ensure that the force field derivative properties are internally consistent with those of the EOS, resulting in better extrapolation for both the EOS and the force field
%\end{enumerate}
%
%\section{Outline: Hybrid data sets for mixtures}
%
%\begin{enumerate}
%	\item Background to establish need for better mixture models
%	\begin{enumerate}
%		\item Design of efficient and reliable chemical processes requires accurate mixture properties over a wide range of temperatures and pressures
%		\item The quantity and quality of experimental data is insufficient for most mixtures
%		\item Molecular simulation can supplement experimental data
%	\end{enumerate}
%	\item Reliability of hybrid data set approach depends on accuracy of force field, specifically, the nonbonded cross interactions are of great importance for mixtures
%	\begin{enumerate}
%		\item Lorentz-Berthelot combining rules are common but not infallible
%		\item Alternative combining rules have not been tested extensively
%		\item Binary interaction parameters have been shown to significantly improve predictability of mixtures of two-center Lennard-Jones plus dipole and quadrupole models 
%	\end{enumerate}
%    \item Since the arithmetic mean for the size parameter, $\sigma_{ij} = \frac{\sigma_i + \sigma_j}{2}$, is often quite reliable, we propose that a binary interaction parameter, $\xi$, be used only to scale $\epsilon_{ij} = \xi \sqrt{\epsilon_i \epsilon_j}$ 
%    \item The methodology proposed in this study is to:
%    \begin{enumerate}
%    	\item Fit Helmholtz free energy EOS to available mixture data
%    	\item Optimize $\xi$ to agree with EOS
%    	\item Iterate:
%    	\begin{enumerate}
%    		\item Fit Helmholtz free energy EOS to hybrid data set
%    		\item Reoptimize $\xi$ (and potentially other force field parameters) to agree with EOS (by using MBAR no additional simulations are required)
%    	\end{enumerate}
%    \end{enumerate}
%%	\item The higher dimensional parameterization when explicitly fitting cross interactions is computationally infeasible when implementing traditional iterative force field optimization approaches
%%	\begin{enumerate}
%%		\item Multistate Bennett Acceptance Ratio (MBAR) combined with basis functions (BF) has been shown to greatly accelerate force field parameterization
%%		\item MBAR reweights configurations sampled with a reference force field to predict thermodynamic properties for a non-simulated force field
%%		\item Basis functions store the contributions from the $r^{-m}$ terms for the nonbonded energies and forces
%%	\end{enumerate}
%    \item With this iterative approach, it is possible to predict more accurate mixture properties
%\end{enumerate}

\section{Description on website}

The current state of research should first be briefly described and supported by approximately five revelant publications from the research area (one page max).

The outline should focus on a clear description of the questions you intend to address in your research, their originality and significance for the advancement of the research field (approx. two pages).

Furthermore, the academic methods to be used to achieve these goals should be clearly described and referenced, if appropriate (approx. two pages).

The research outline should comprise approximately five pages in total (including references). Should you significantly exceed this length, you may be asked to cut it down to approximately five pages.

For the purposes of evaluation it must be clearly demonstrated that you yourself have drawn up the main contents independently and agreed them beforehand with your host. Any contents contributed by the host institute must be attributed accordingly.

\section{Rough draft: Developing more accurate nonbonded potentials for EOS development}

\subsection{Introduction}

As most chemical processes contain mixtures of several molecular species, the design of efficient and reliable chemical processes requires accurate prediction of thermophysical properties over a wide range of temperatures, pressures, and compositions. 

 Furthermore, as most chemical processes contain mixtures of several species, mixture data are extremely valuable and important for the design of chemical processes.  equations of state over a wide range of temperatures and pressures.

The ost chemical processes contain mixtures of several species, mixture data are extremely valuable and important for the design of chemical processes. Unfortunately, mixture data are especially scarce and the reliability is even more dubious than pure component data.

%The design of efficient and reliable chemical processes requires accurate equations of state over a wide range of temperatures and pressures.

The design of efficient and reliable chemical processes requires accurate prediction of thermophysical properties over a wide range of temperatures and pressures. Fundamental equations of state (FEOS), such as those based on the Helmholtz free energy (e.g. REFPROP), are a powerful approach for estimating pressure, density, temperature $(P\rho T)$ behavior and caloric properties, such as internal energy $(U)$ and isochoric/isobaric heat capacities $(c_{\rm v}$ and $c_{\rm p}$, respectively). Unfortunately, most compounds do not have sufficient \textit{reliable} experimental data for a diverse set of thermodynamic properties covering a wide range of $P \rho T$ conditions to develop a highly-accurate FEOS. The lack of experimental data at high temperatures and pressures, especially, is attributed to the inherent safety, cost, and complexity of such experiments. 

As most chemical processes contain mixtures of several species, reliable estimation of mixture properties is extremely important. Although the FEOS is capable of predicting mixture properties, typically the deviations are larger than the pure species FEOS. One reason for the deficiency in mixture FEOS models is the scarcity of reliable mixture data. For this reason, applying the hybrid data set approach to mixtures is a potentially promising avenue. Developing reliable force fields and FEOS for pure species is an essential first step before focusing on multi-component systems.

% molecular simulation of mixture properties

% Unf mixture data are especially scarce mixture data are extremely valuable important for the design of chemical processes. Unfortunately, mixture data are especially scarce and the reliability is even more dubious than pure component data. 

% but extremely valuable and important for the design of chemical processes. 

%Mixture data are even more scarce
%Need accurate pure species models before we can tackle mixtures

By contrast, molecular simulation (i.e. Monte Carlo, MC, and molecular dynamics, MD) methods at high temperatures and pressures do not suffer from any of these limitations. For this reason, molecular simulation results at extreme temperatures and pressures can supplement experimental data \cite{Thol2016_LJ,Thol_LJTS,Rutkai2017,Lustig2015,Rutkai2015}. Recently, the host institute successfully applied this ``hybrid data set'' approach to dramatically extend the FEOS' range of applicability for several compounds \cite{Rutkai2013,Thol2016_siloxane_first,Thol2016_siloxane,Thol2017,Thol2015}. The inclusion of these simulation results improved the performance of the FEOS at extreme temperatures and pressures. To improve the performance of this so-called ``hybrid data set'' approach, it is imperative that the force field be transferable over different $P \rho T$ conditions. In a previous study, we demonstrated the poor transferability to high pressures of the popular united-atom Mie $n$-6 potential (of which the traditional Lennard-Jones 12-6 is a subclass). Therefore, we propose that alternative potential models be considered for developing FEOS. Specifically, we propose the development of extended Lennard-Jones (ex-LJ) force fields to improve the fundamental equations of state at high pressures.

The primary limitation of this approach is that most force fields are not ``transferable'' over a wide range of $P \rho T$ conditions. For example, in a previous study, we demonstrated the poor transferability to high pressures of the popular united-atom Mie $n$-6 potential (of which the traditional Lennard-Jones 12-6 is a subclass).    

\subsection{Theory}

%Fundamental equations of state based on the Helmholtz free energy (HFE-FEOS) allow for reliable prediction of 

Fundamental equations of state based on the Helmholtz free energy (HFE-FEOS) are expressed as 
\begin{equation}
\frac{A}{R_gT} \equiv \alpha = \alpha^{\rm ig} + \alpha^{\rm r}
\end{equation} 
where $A$ is the Helmholtz free energy, $R_g$ is the gas constant, $T$ is temperature, $\alpha$ is reduced Helmholtz free energy, $\alpha^{\rm ig}$ and $\alpha^{\rm r}$ are the ideal gas and residual contributions to the reduced Helmholtz free energy. As $\alpha^{\rm ig}$ is typically obtained using first principles (\textit{ab initio}) calculations, the primary focus of FEOS development is modeling $\alpha^{\rm r}$. The state-of-the-art model for pure species is  
\begin{equation} \label{eq: Residual Helmholtz}
\alpha^{\rm r} = \alpha_{\rm Pol}^{\rm r} + \alpha_{\rm Exp}^{\rm r} + \alpha_{\rm GBS}^{\rm r}
\end{equation}
Note that Equation \ref{eq: Residual Helmholtz} is semi-empirical and has a large number of fitting parameters (between 50 and 100). To avoid over-fitting, the number of parameters in Equation \ref{eq: Residual Helmholtz} depends on the amount of experimental data used to train the FEOS. The FEOS predictions can result in large errors when extrapolated to higher temperatures and pressures than the conditions for the training data. Improvement in an FEOS at high temperatures and pressures necessitates additional data for those conditions.

%The ideal gas contribution to the Helmholtz free energy, $\alpha^{\rm ig}$, is typically obtained using first principles (\textit{ab initio}) calculations. The residual contribution, $\alpha^{\rm r}$, is the primary focus of FEOS development. The state-of-the-art approach is to use a highly flexible model  
%\begin{equation} \label{eq: Residual Helmholtz}
%\alpha^{\rm r} = \alpha_{\rm Pol}^{\rm r} + \alpha_{\rm Exp}^{\rm r} + \alpha_{\rm GBS}^{\rm r}
%\end{equation}
%Note that Equation \ref{eq: Residual Helmholtz} is semi-empirical and has a large number of fitting parameters (between 50 and 100). More parameters are permissible if the experimental data used to train the FEOS cover a wide enough range of temperatures and pressures. Therefore, the FEOS predictions can result in large errors when extrapolated to higher temperatures and pressures than the conditions for the training data. Improvement in an FEOS at high temperatures and pressures necessitates additional data for those conditions. 

%To avoid over-fitting, it is important that the number of model parameters 

The primary advantage of HFE-FEOS is that all thermodynamic properties are related to some combination of Helmholtz free energy derivatives with respect to temperature and density. The main disadvantage of HFE-FEOS is that large amounts of reliable experimental data are required which cover a wide range of temperatures and pressures. 
%??=??^o+??^r
%??^r=??_Pol^r+??_Exp^r+??_GBS^r

For example, the following expressions demonstrate the interdependency of several thermodynamic properties \cite{Rutkai2015}:
\begin{equation} \label{eq:Z}
Z = 1 + A_{01}^{\rm dep}
\end{equation}
\begin{equation} \label{eq:Z_IC}
\frac{1}{T} \left(\frac{-\partial Z}{\partial(1/T)}\right)_\rho = 1 + A_{01}^{\rm dep} - A_{11}^{\rm dep}
\end{equation}
\begin{equation} \label{eq:U}
\frac{U^{\rm dep}}{R_gT} = A_{10}^{\rm dep}
\end{equation}
\begin{equation} \label{eq:U_IC}
\frac{1}{R_g} \left(\frac{\partial U^{\rm dep}}{\partial T}\right)_\rho = -A_{20}^{\rm dep} 
\end{equation} 
\begin{equation} \label{eq:Z_IT}
\rho \left(\frac{\partial Z}{\partial \rho}\right)_T = 1 + 2 A_{01}^{\rm dep} + A_{02}^{\rm dep} 
\end{equation}
where we introduce the notation 
\begin{equation} \label{Residual Derivatives}
A_{xy}^{\rm dep} R_gT \equiv (1/T)^x\rho^y \frac{\partial^{x+y}A^{\rm dep}}{\partial(1/T)^x \partial \rho^y}
\end{equation}

The development of fundamental equations of state (FEOS) based on the Helmholtz free energy is by no means novel. However, the use of hybrid data sets consisting of experimental and molecular simulation values has only been implemented in the last five years. 

The simulation values that are typically included in hybrid data sets are derivatives of the residual Helmholtz free energy with respect to inverse temperature and/or density \cite{Thol2016_siloxane_first,Thol2016_siloxane,Thol2017,Rutkai2013,Thol2015}:
\begin{equation} \label{Residual Derivatives}
A_{xy}^{\rm r} R_gT \equiv (1/T)^x\rho^y \frac{\partial^{x+y}A^{\rm r}}{\partial(1/T)^x \partial \rho^y}
\end{equation}
Note that \textit{ms2} \cite{ms2}, the open-source simulation package developed by Jadran Vrabec's group, already computes $A_{xy}^{\rm r}$. Therefore, the infrastructure is already in place to implement the hybrid data set approach at the Technische Universitat Berlin.
   
We would like to emphasize the advantage of using $A_{xy}^{\rm dep}$ for developing FEOS, as this approach eliminates redundant information found in traditional macroscopic properties \cite{Thol2016_LJ,Rutkai2017,Lustig2015,Rutkai2013,Rutkai2015}. Furthermore, experimental data are typically measured for properties that relate only to first and second derivatives. By contrast, in principle, molecular simulation provides an avenue for estimating third derivatives and higher order terms, which are valuable when fitting FEOS. In addition, with the iterative hybrid data set approach Equation BLANK can be used to provide non-redundant information for fitting the non-bonded potential parameters. 
   
In principle, this process could be reversed such that Equation BLANK would provide non-redundant information for fitting the non-bonded potential parameters.
   
We would like to emphasize the advantage of using $A_{xy}^{\rm dep}$ for developing FEOS, as this approach eliminates redundant information found in traditional macroscopic properties \cite{Thol2016_LJ,Rutkai2017,Lustig2015,Rutkai2013,Rutkai2015}. For example, the following expressions demonstrate the interdependency of several thermodynamic properties \cite{Rutkai2015}:
\begin{equation} \label{eq:Z}
Z = 1 + A_{01}^{\rm dep}
\end{equation}
\begin{equation} \label{eq:Z_IC}
\frac{1}{T} \left(\frac{-\partial Z}{\partial(1/T)}\right)_\rho = 1 + A_{01}^{\rm dep} - A_{11}^{\rm dep}
\end{equation}
\begin{equation} \label{eq:U}
\frac{U^{\rm dep}}{R_gT} = A_{10}^{\rm dep}
\end{equation}
\begin{equation} \label{eq:U_IC}
\frac{1}{R_g} \left(\frac{\partial U^{\rm dep}}{\partial T}\right)_\rho = -A_{20}^{\rm dep} 
\end{equation} 
\begin{equation} \label{eq:Z_IT}
\rho \left(\frac{\partial Z}{\partial \rho}\right)_T = 1 + 2 A_{01}^{\rm dep} + A_{02}^{\rm dep} 
\end{equation}

The Lennard-Jones 12-6 potential is limited



For example, the 12-10-8-6 ex-LJ potential is expressed as 
\begin{equation}
U_{\rm ex-LJ} = C_{12} r^{-12} + C_{10} r^{-10} + C_{8} r^{-8} + C_{6} r^{-6}
\end{equation}
where $C_{i}$ is the coefficient for the $r^{-i}$ term. Note that there must be at least one positive and one negative coefficient to account for the repulsive and attractive contributions, respectively. By solving a system equations, the coefficients can be related to the more traditional non-bonded parameters, namely, $\epsilon$, $\sigma$, and $r_{\rm min}$:
\begin{eqnarray}
U_{\rm ex-LJ}(r_{\rm min}) = -\epsilon
\end{eqnarray}
\begin{eqnarray}
U_{\rm ex-LJ}(\sigma) = 0
\end{eqnarray}
\begin{eqnarray}
\frac{\partial U_{\rm ex-LJ}(r_{\rm min})}{\partial r} = 0
\end{eqnarray}

Also, by defining $\epsilon$, $\sigma$, and $r_{\rm min}$

The most general expression for the extended Lennard-Jones non-bonded potential is:
\begin{equation} \label{ex-LJ general}
u_{\rm nb, exLJ} = \sum_m C_m r^{-m} 
\end{equation} 
where $m$ are integer values and $C_m$ are the coefficients for the corresponding $r^{-m}$ terms. Note that the traditional Lennard-Jones 12-6 potential is obtained if $C_6$ and $C_{12}$ are negative and positive, respectively, while all other $C_m$ values are zero. Traditionally, there are only two to four non-zero $C_m$ values, e.g., 16-6, 12-8-6, 14-12-8-6. The primary reasons for this are to simplify the fitting and to avoid over-fitting if the data are insufficient to characterize more parameters. In principle, however, any number of $C_m$ terms could be non-zero. 

The primary deficiency in the Mie $\lambda$-6 potential is that the optimal value of $\lambda$ is greater than 12 for reproducing vapor-liquid equilibria properties. Unfortunately, this conclusion is primarily  

Improved extrapolation towards high pressure requires a more physically realistic non-bonded potential. 

The extended Lennard-Jones potential 

The ex-LJ 12-10-8-6 potential is more flexible than the two-parameter ($\epsilon$ and $\sigma$) LJ 12-6, and more theoretically justified than the three-parameter Mie $n$-6, particularly when $n \gg 12$. However, it has not been tested as extensively as the LJ 12-6, Mie $n$-6, and exponential-6 potentials. Although the ex-LJ was proposed over two decades previously, the main reason for the lack of popularity is the additional complexity in parameterizing the ex-LJ potential. Due to the additional model parameters, the traditional approach

The $r^{-6}$, $r^{-8}$, and $r^{-10}$ terms can be derived from dispersive interactions and, therefore, it seems most reasonable for $C_6$, $C_8$, and $C_{10}$ to be negative. However, it is unclear how well a 12-10-8-6 potential would perform if only the $C_{12}$ coefficient is positive. Therefore, from an empirical standpoint, it is possible to allow all coefficients to be either positive or negative.

Whereas the LJ 12-6, Mie $n$-6, and expo

For example, the traditional approach to parameterize the Mie $n$-6 potential is to compare experimental vapor-liquid equilibria data with molecular simulation values for different sets of $\epsilon$, $\sigma$, and $n$. For the ex-LJ potential, this brute-force approach is computationally infeasible. For this reason, we propose an alternative optimization scheme that was recently developed.

higher dimensional parameterization of the ex-LJ is computationally demanding when the force field is parameterized using traditional brute-force methods.

One reason for the lack of popularity is the increased complexity due to the higher dimensional parameterization.

A large amount of experimental data are required to obtain a unique set of non-bonded parameters. Fitting non-bonded parameters to the Helmholtz free energy derivatives can provide non-redundant information.



To facilitate parameterization of ex-LJ potentials, we propose the use of Multistate Bennett Acceptance Ratio (MBAR) combined with basis functions (BF). Previous publications demonstrate that MBAR-BF reduces the computational cost by several orders of magnitude compared to direct molecular simulation. The combined MBAR-BF approach estimates ensemble averages for   MBAR has been shown to be a reliable approach for reweighting conf has been shown to greatly accelerate force field parameterization

MBAR reweights configurations that are sampled with a reference force field to predict thermodynamic properties for a non-simulated force field. MBAR requires ``recalculating'' the non-bonded energies and forces for each configuration sampled. Basis functions greatly accelerate the cost of this recalculation step by storing the $\sum r^{-m}$ contributions. Basis functions are amenable to the extended Lennard-Jones potential. For example, the 

Basis functions store the contributions from the $r^{-m}$ terms for the nonbonded energies and forces
  
\subsection{Methods}
  
  
  
The methodology proposed in this study is to:
\begin{enumerate}
	\item Develop a FEOS over the temperatures and pressures where reliable experimental data exist
	\item Fit the ex-LJ parameters to FEOS derivative Helmholtz free energy properties in range where FEOS is reliable
	\item Perform molecular simulations at state points where experimental data are not available
	\item Iterate:
	\begin{enumerate}
		\item Refit the FEOS to the hybrid data set
		\item Re-optimize the force field parameters to match FEOS (using MBAR so that no additional molecular simulations are required)
	\end{enumerate}
\end{enumerate}

With this iterative approach, it is possible to ensure that the Helmholtz free energy derivative properties are internally consistent between the FEOS and the ex-LJ force field. The purpose for this iterative approach is to improve the extrapolation of FEOS and the reliability of the force field. Also, as the FEOS and force field become more internally consistent with each successive iteration, it is possible to increase the number of fitting parameters in both the FEOS and in the ex-LJ. 

With this iterative approach, it is possible to ensure that the force field derivative properties are internally consistent with those of the FEOS, resulting in better extrapolation for both the EOS and the force field. 

As the ex-LJ potential has not been studied extensively, a significant step in this proposal is to develop new combining rules for cross interactions. For example, applying the traditional Lorentz-Berthelot combining rules allows for estimation of $\epsilon_{ij}$, $\sigma_{ij}$, and/or $r_{\rm min,ij}$ from $\epsilon_{ii}$, $\sigma_{ii}$, and/or $r_{\rm min,ii}$. Even in pure species systems, cross interactions are typically implemented to simplify the parameterization and to avoid non-physical parameters. The cross interaction parameters play a key role in mixture property estimates.  

For example, the traditional Lorentz-Berthelot combining rules allow for estimation of $\epsilon_{ij}$, $\sigma_{ij}$, and/or $r_{\rm min,ij}$ from the . 

The focus of this research is pure species properties 

Important to develop new combining rules for cross interactions of ex-LJ. Additional binary interaction parameters improve agreement with mixture properties

Although the focus of this research is developing highly accurate FEOS for pure species, it should be noted that reliable estimation of mixture properties is an extremely important area.



%Specifically, the optimal $\lambda$ value is greater than $12$ for vapor-liquid equilibria which results in significant over-prediction of pressure.

%Modern force fields for small compounds are sufficiently reliable at extreme state points to supplement experimental data when developing FEOS. By contrast, popular united-atom force fields for normal and branched alkanes that are highly accurate for estimating vapor-liquid equilbiria (VLE) properties suffer from significant systematic deviations at high pressures for the compressibility factor $(Z)$, $U$, and $c_{\rm v}$. For example, Bayesian inference suggests that the UA Mie $\lambda$-6 model type is not adequate for simultaneously predicting saturated liquid density $(\rho_{\rm l}^{\rm sat})$, saturated vapor pressure $(P_{\rm v}^{\rm sat})$, $Z$, $U$, and $c_{\rm v}$. Therefore, while considerable improvement in VLE is observed for the state-of-the-art Mie $\lambda$-6 potential over the traditional Lennard-Jones 12-6, the results from this study suggest that alternative models be considered for developing FEOS of normal and branched alkanes, such as force fields that use anisotropic-united-atom, all-atom, and/or alternative non-bonded potentials.

%Molecular simulation results at extreme temperatures and pressures can supplement experimental data when developing fundamental equations of state. Since most force fields are optimized to agree with vapor-liquid equilibria (VLE) properties, however, the reliability of the molecular simulation results depends on the validity/transferability of the force field at higher temperatures and pressures. As demonstrated in this study, although state-of-the-art united-atom Mie $\lambda$-6 potentials for normal and branched alkanes provide accurate estimates for VLE, they tend to over-predict pressures for dense supercritical fluids and compressed liquids. The physical explanation for this observation is that the repulsive barrier is too steep for the ``optimal'' united-atom Mie $\lambda$-6 potential parameterized with VLE properties. Bayesian inference confirms that no feasible combination of non-bonded parameters ($\epsilon$, $\sigma$, and $\lambda$) is capable of simultaneously predicting saturated vapor pressures, saturated liquid densities, and pressures at high temperatures and densities. This conclusion has both practical and theoretical ramifications, as more realistic non-bonded potentials may be required for accurate extrapolation to high pressures of industrial interest.

\section{Outline: Developing more accurate nonbonded potentials for EOS development}

\begin{enumerate}
	\item Background to establish need for hybrid data sets
	\begin{enumerate}
		\item Design of efficient and reliable chemical processes requires accurate equations of state over a wide range of temperatures and pressures
		\item The quantity and quality of experimental data is insufficient for most compounds
		\item Molecular simulation is an ideal approach for overcoming the data deficiency at extreme temperatures and pressures
		\item Improved EOS are developed by supplementing experimental data with simulation results
	\end{enumerate}
	\item Reliability of hybrid data set approach depends on accuracy of force field, specifically, how well the model extrapolates to extreme temperatures and pressures
	\begin{enumerate}
		\item Recent work has demonstrated that the united-atom Lennard-Jones $n$-6 force field does not extrapolate from vapor-liquid equilibria conditions to elevated pressures
		\item Improved extrapolation requires a more physically realistic nonbonded potential
		\item For this reason, we propose the use of an extended Lennard-Jones (ex-LJ, 12-10-8-6) potential
	\end{enumerate}
	\item The ex-LJ 12-10-8-6 potential is more flexible than the LJ 12-6, and more theoretically justified than the Mie $n$-6
	\item Although the ex-LJ was proposed over two decades previously, it has not been tested as extensively as the LJ 12-6, Mie $n$-6, and exponential-6
	\item One reason for the lack of popularity is that the higher dimensional parameterization of the ex-LJ is computationally infeasible when implementing traditional iterative force field optimization approaches
	\begin{enumerate}
		\item Multistate Bennett Acceptance Ratio (MBAR) combined with basis functions (BF) has been shown to greatly accelerate force field parameterization
		\item MBAR reweights configurations sampled with a reference force field to predict thermodynamic properites for a non-simulated force field
		\item Basis functions store the contributions from the $r^{-m}$ terms for the nonbonded energies and forces
	\end{enumerate}
	\item The methodology proposed in this study is to:
	\begin{enumerate}
		\item Develop an EOS over the temperatures and pressures where reliable experimental data exist
		\item Fit the ex-LJ parameters to EOS derivative Helmholtz energy properties in range where EOS is reliable
		\item Perform molecular simulations at state points where experimental data are not available
		\item Iterate:
		\begin{enumerate}
			\item Refit the EOS to the hybrid data set
			\item Reoptimize the force field parameters to match EOS (using MBAR so that no additional molecular simulations are required)
		\end{enumerate}
	\end{enumerate}
	\item With this iterative approach, it is possible to ensure that the force field derivative properties are internally consistent with those of the EOS, resulting in better extrapolation for both the EOS and the force field
    \item Important to develop new combining rules for cross interactions of ex-LJ
    \item Additional binary interaction parameters improve agreement with mixture properties
\end{enumerate}

\section{Justification for funding}
\begin{enumerate}
    \item Why is this important?
    \begin{enumerate}
	    \item Reliable estimates of thermophysical properties have signficant practical industrial applications
    \end{enumerate}	
    \item Why me?
    \begin{enumerate}
	    \item Expertise in force field parameterization and, specifically, the MBAR-BF method
	    \item Expertise in both molecular dynamics and Monte Carlo simulations
	\end{enumerate}
	\item Novel aspects of this research to expand my expertise:
	\begin{enumerate}
		\item Fitting fundamental equations of state
		\item Residual helmholtz energy derivatives from molecular simulation
		\item Hybrid data set approach
	\end{enumerate}
    \item Why Jadran?
    \begin{enumerate}
	    \item Dr. Vrabec has helped pioneer the FEOS hybrid data set approach
	    \item Substantial computational resources at disposal
	    \item Strong research group and collaborators
	    \item Experience developing molecular simulation software
    \end{enumerate}
\end{enumerate}

%The reliability of molecular simulation results depends primarily on the force field. In particular, several thermophysical properties are highly sensitive to the non-bonded interactions. Unfortunately, determining the ``optimal'' parameters for a non-bonded potential model has traditionally been an arduous and time-consuming task. For example, non-bonded potentials are frequently parameterized using vapor-liquid equilibria (VLE) properties, such as saturated liquid and vapor densities, saturated vapor pressures, and heat of vaporization. This parameterization approach often requires large amounts of Gibbs Ensemble Monte Carlo (GEMC) or Grand Canonical Monte Carlo (GCMC) simulations using hundreds of different non-bonded parameter values. 
%
%A more efficient approach than simulating each proposed non-bonded parameter set is to reuse the information gained from nearby parameter sets. This study demonstrates how to utilize Multistate Bennett Acceptance Ratio (MBAR) for this purpose. Specifically, MBAR reweights configurations that are sampled using a few reference parameter sets to predict physical property values for other non-bonded parameter sets, without performing additional direct simulations. 
%
%The MBAR algorithm requires that the energies are recomputed for the sampled configurations using the new non-bonded parameter set. Basis functions are shown to be a computationally efficient method for recomputing these energies. Basis functions eliminate the need for looping through non-bonded interactions by providing a linear relationship between the total non-bonded energy and the non-bonded parameters. 
%
%We demonstrate that MBAR with basis functions reduces the computational cost by approximately three to five orders of magnitude relative to direct GEMC and GCMC simulations, where the reduction is greatest for larger molecules and systems. With this significant computational speed-up, we use a robust, high-dimensional, Bayesian optimization routine to simultaneously parameterize a Mie n-6 potential for several interaction sites.
%		
%	Over the past decade, the Mie $n$-6 (generalized Lennard-Jones, LJ) non-bonded potential has provided significant improvement when predicting vapor-liquid equilibria (VLE) properties of organic compounds. For united-atom (UA) force fields parameterized with VLE data, the optimal value of $n$ is typically greater than 12, in contrast to the traditional LJ 12-6. However, there exist strong theoretical concerns that $n \ge 12$ is too repulsive at short distances. While the forces at close-range distances do not dramatically impact VLE properties, they can play a large role at high pressures.
%		
%	For this reason, we investigate the practical implications of using a Mie $n$-6 potential for $n\ge12$ at high pressures. Specifically, we determine if the UA Mie $n$-6 accurately predicts the compressibility factor $(Z)$ and viscosity $(\eta)$ of normal and branched alkanes at high pressures. We observe a large positive bias in $Z$ and $\eta$ for $n>12$ at high pressures that increases with increasing $n$. Bayesian inference of the non-bonded parameters demonstrates that no set of $\epsilon$, $\sigma$, and $n$ adequately predicts both VLE and high pressure properties. Also, we do not observe any improvement when comparing the Mie $n$-6 results with those of the Buckingham exponential-6 potential, which is purported to have a more realistic repulsive barrier. These observations are of both practical and theoretical significance when selecting the ``best'' function form for computing non-bonded interactions.
%	
%	The reliability of molecular simulation results depends primarily on the force field, with the non-bonded interactions playing a key role. Unfortunately, non-bonded potentials that are empirically parameterized with thermophysical properties traditionally use iterative brute-force methods that require large amounts of molecular simulation. 
%
%A more efficient approach than simulating each proposed non-bonded parameter set is to reuse the information gained from nearby parameter sets. This study demonstrates how to utilize Multistate Bennett Acceptance Ratio (MBAR) for this purpose. Specifically, MBAR reweights configurations that are sampled using a few reference parameter sets to predict the internal energy and pressure for other non-bonded parameter sets, without performing additional direct simulations. 
%
%The MBAR algorithm requires that the energies and forces are recomputed for the sampled configurations using the new non-bonded parameter set. Basis functions are shown to be a computationally efficient method for recomputing these quantities. Basis functions eliminate the need for looping through non-bonded interactions by providing a linear relationship between the total non-bonded energy and the non-bonded parameters. 
%
%We demonstrate how to implement MBAR with NVT isothermal isochoric integration (ITIC) and Grand Canonical Monte Carlo (GCMC) to obtain vapor-liquid equilibria properties. MBAR with basis functions reduces the computational cost by approximately three to five orders of magnitude relative to direct simulation of VLE, where the reduction is greatest for larger molecules and systems. With this significant computational speed-up, we use a robust, high-dimensional, Bayesian optimization routine to simultaneously parameterize a Mie $n$-6 potential for several interaction sites. 
%
%	Accurate prediction of viscosity $(\eta)$ at high pressures $(P)$ necessitates an extremely reliable force field for at least two reasons. First, the viscosity at a given density $(\rho)$ is highly sensitive to the function form and associated parameters, especially those of the non-bonded interactions. Second, the viscosity depends strongly on the predicted density, which is also very sensitive to the force field. 
%
%To develop a highly accurate force field, the CH$_3$, CH$_2$, CH, and C non-bonded parameters are optimized simultaneously using a large data set. Since the challenge compound is 2,2,4-trimethylhexane (TMH), we chose 2,2,4-trimethylpentane (TMP, a.k.a. isooctane) as a surrogate molecule to improve transferability. Specifically, the optimal non-bonded parameters are obtained empirically by minimizing the deviation between the predicted and REFPROP $P\rho T$ and caloric properties for TMP over a wide range of state points, with particular emphasis at high pressures. 
%
%An essential aspect of the challenge is to provide meaningful estimates of uncertainty. For this reason, uncertainties in the predicted TMH viscosity are quantified using three different methods. First, we estimate systematic bias in the force field by comparing the simulated and REFPROP $\eta$ values for the surrogate compound, TMP, at the challenge temperature and pressures. Second, we account for the uncertainty in $\eta$ that is associated with uncertainties in $\rho$ for a given $P$. Third, we propagate the uncertainty in the force field non-bonded parameters using Bayesian inference. 


%Fundamental equations of state (FEOS), such as those based on the Helmholtz free energy (e.g. REFPROP), are a powerful approach for estimating pressure, density, temperature $(P\rho T)$ behavior and caloric properties, such as internal energy $(U)$ and isochoric/isobaric heat capacities $(c_{\rm v}$ and $c_{\rm p}$, respectively). Unfortunately, most compounds do not have sufficient \textit{reliable} experimental data for a diverse set of thermodynamic properties covering a wide range of $P \rho T$ conditions to develop a highly-accurate FEOS. The lack of experimental data at high temperatures and pressures, especially, is attributed to the inherent safety, cost, and complexity of such experiments. By contrast, molecular simulation (i.e. Monte Carlo, MC, and molecular dynamics, MD) methods at high temperatures and pressures do not suffer from any of these limitations.
%
%For this reason, FEOS are developed for compounds with limited experimental data by including molecular simulation results at elevated temperatures and pressures. For this so-called ``hybrid data set'' approach to work, it is imperative that the force field be transferable over different $P \rho T$ conditions. Modern force fields for small compounds are sufficiently reliable at extreme state points to supplement experimental data when developing FEOS. By contrast, popular united-atom force fields for normal and branched alkanes that are highly accurate for estimating vapor-liquid equilbiria (VLE) properties suffer from significant systematic deviations at high pressures for the compressibility factor $(Z)$, $U$, and $c_{\rm v}$. For example, Bayesian inference suggests that the UA Mie $\lambda$-6 model type is not adequate for simultaneously predicting saturated liquid density $(\rho_{\rm l}^{\rm sat})$, saturated vapor pressure $(P_{\rm v}^{\rm sat})$, $Z$, $U$, and $c_{\rm v}$. Therefore, while considerable improvement in VLE is observed for the state-of-the-art Mie $\lambda$-6 potential over the traditional Lennard-Jones 12-6, the results from this study suggest that alternative models be considered for developing FEOS of normal and branched alkanes, such as force fields that use anisotropic-united-atom, all-atom, and/or alternative non-bonded potentials.

%Molecular simulation results at extreme temperatures and pressures can supplement experimental data when developing fundamental equations of state. Since most force fields are optimized to agree with vapor-liquid equilibria (VLE) properties, however, the reliability of the molecular simulation results depends on the validity/transferability of the force field at higher temperatures and pressures. As demonstrated in this study, although state-of-the-art united-atom Mie $\lambda$-6 potentials for normal and branched alkanes provide accurate estimates for VLE, they tend to over-predict pressures for dense supercritical fluids and compressed liquids. The physical explanation for this observation is that the repulsive barrier is too steep for the ``optimal'' united-atom Mie $\lambda$-6 potential parameterized with VLE properties. Bayesian inference confirms that no feasible combination of non-bonded parameters ($\epsilon$, $\sigma$, and $\lambda$) is capable of simultaneously predicting saturated vapor pressures, saturated liquid densities, and pressures at high temperatures and densities. This conclusion has both practical and theoretical ramifications, as more realistic non-bonded potentials may be required for accurate extrapolation to high pressures of industrial interest. 

%This section discusses limitations to the primary conclusion from this study, namely, that UA Mie $\lambda$-6 force fields parameterized with VLE data should not be used to develop fundamental equations of state for normal and branched alkanes. The main limitation is that the poor extrapolation at high pressures is based solely on the trend of $Z$ with respect to inverse temperature. By contrast, the simulation values that are typically included in hybrid data sets are derivatives of the residual (or departure) Helmholtz free energy with respect to inverse temperature and/or density \cite{Thol2016_siloxane_first,Thol2016_siloxane,Thol2017,Rutkai2013,Thol2015}:
%\begin{equation} \label{Residual Derivatives}
%A_{xy}^{\rm dep} R_gT \equiv (1/T)^x\rho^y \frac{\partial^{x+y}A^{\rm dep}}{\partial(1/T)^x \partial \rho^y}
%\end{equation}
%We would like to emphasize the advantage of using $A_{xy}^{\rm dep}$ for developing FEOS, as this approach eliminates redundant information found in traditional macroscopic properties \cite{Thol2016_LJ,Rutkai2017,Lustig2015,Rutkai2013,Rutkai2015}. For example, the following expressions demonstrate the interdependency of the properties we computed, namely, $Z$ and $U^{\rm dep}$ with their derivatives along isochores and isotherms \cite{Rutkai2015}:
%\begin{equation} \label{eq:Z}
%Z = 1 + A_{01}^{\rm dep}
%\end{equation}
%\begin{equation} \label{eq:Z_IC}
%\frac{1}{T} \left(\frac{-\partial Z}{\partial(1/T)}\right)_\rho = 1 + A_{01}^{\rm dep} - A_{11}^{\rm dep}
%\end{equation}
%\begin{equation} \label{eq:U}
%\frac{U^{\rm dep}}{R_gT} = A_{10}^{\rm dep}
%\end{equation}
%\begin{equation} \label{eq:U_IC}
%\frac{1}{R_g} \left(\frac{\partial U^{\rm dep}}{\partial T}\right)_\rho = -A_{20}^{\rm dep} 
%\end{equation} 
%\begin{equation} \label{eq:Z_IT}
%\rho \left(\frac{\partial Z}{\partial \rho}\right)_T = 1 + 2 A_{01}^{\rm dep} + A_{02}^{\rm dep} 
%\end{equation}
%Unfortunately, with the exception of \textit{ms2} \cite{ms2}, we are not aware of any open-source simulation package that readily provides $A_{xy}^{\rm dep}$. 
%%MRS2: this is something we can set up relatively easily for OpenFF (along with the similar derivatives for NPT), and perhapsshould.  Owen has been working along these lines.
%%RAM2: I assume you mean OpenMM?
%
%In addition, macroscopic properties, such as $Z$ and $U^{\rm dep}$ (with their respective derivatives), are more readily understood and visualized than $A_{xy}^{\rm dep}$. Also, it is easier to quantify their impact on process design than derivatives in the residual Helmholtz free energy. For example, as demonstrated in 
%%MRS3: check if there is a way to cite the number not superscripted.
%%RAM3: Should have used \citenum not \cite
%Reference \citenum{Thol2016_siloxane_first}, an inaccurate prediction of some $A_{xy}^{\rm dep}$ does not necessarily result in poor prediction of $P \rho T$ behavior or heat capacities. For these reasons, we did not base our conclusions on the adequacy of the UA Mie $\lambda$-6 potential to predict $A_{xy}^{\rm dep}$.
%
%Instead, we have presented $Z$ and, by inspection, the slope of $Z$ with respect to $1/T$ at constant $\rho$. Since these properties are equivalent to Equations \ref{eq:Z}-\ref{eq:Z_IC}, respectively, we have indirectly focused on only two of the Helmholtz derivatives, namely, $A_{01}^{\rm dep}$ and $A_{11}^{\rm dep}$. However, we also observed some deviations in $U^{\rm dep}$, the slope of $U^{\rm dep}$ with respect to $T$ at constant $\rho$, and the slope of $Z$ with respect to $\rho$ at constant $T$, which are equivalent to Equations \ref{eq:U}-\ref{eq:Z_IT}. Unfortunately, deficiencies are not as obvious for these properties due to the relatively large uncertainties in the REFPROP correlations, ca. 5~\% and 10~\% for $U^{\rm dep}$ and $\left(\frac{\partial U^{\rm dep}}{R_g \partial T}\right)_\rho$. Although these additional properties provide insight regarding $A_{10}^{\rm dep}$, $A_{20}^{\rm dep}$, $A_{01}^{\rm dep}$, and $A_{02}^{\rm dep}$, because the results are less conclusive they are provided only in Section \ref{Udep and Z_IT} of Supporting Information. Furthermore, the relationship between Equations \ref{eq:U}-\ref{eq:Z_IT} and the repulsive barrier, $\lambda$, is not obvious. By contrast, the observation that larger values of $\lambda$ over-predict $Z$ with increasing temperature is clear evidence that $\lambda > 12$ is not theoretically justified but, rather, is an empirical remedy that performs well for VLE.
%
%Future work should investigate more thoroughly the adequacy of UA Mie $\lambda$-6 (or other) force fields to predict $U^{\rm dep}$ and isochoric/isobaric heat capacities at high temperatures and pressures. For example, rigorous studies of this type have been performed on the single-site Lennard-Jones fluid \cite{Thol2016_LJ,Thol_LJTS,Rutkai2017}.  

Importance sampling is a commonly used statistical technique for computing averages of properties for one model by reweighting configurations sampled with another model based on the ratio of probabilities for the two models. In chemistry and chemical physics, importance sampling from one or a set of simulations to another set of simulation conditions can be implemented using the MBAR algorithm.\cite{shirts-chodera:jcp:2008:mbar,arxivShirts2017}
With MBAR the expectation $\langle O(\theta) \rangle$ for force field $(\theta)$ of any given observable $(O)$ can be expressed as:
\begin{eqnarray} \label{MBAR O theta}
\langle O(\theta)\rangle = \sum_{n=1}^{N} O(\x_{n};\theta) W_{n}(\theta)
\end{eqnarray}
where $\x_n$ are configurations sampled from one or more reference force fields $(\theta_{\rm ref})$, 
$O(\x_{n};\theta)$ is the observable value using force field 
$\theta$ with configurations $\x_n$, and $W_{n}(\theta)$ is the weight of the $n^{th}$
configuration using force field $\theta$, calculated by using:
\begin{eqnarray} \label{MBAR weights theta}
W_{n}(\theta) = \frac{\exp[\hat{f}(\theta)-u(\x_{n};\theta)]}{\sum\limits_{k=1}^K N_k \, \exp[\hat{f}(\theta_{\rm ref,k}) - u(\x_{n};\theta_{\rm ref,k})]}
\end{eqnarray}
where the reduced free energies $(\hat f(\theta))$ are calculated with:
\begin{eqnarray} \label{MBAR free energy theta}
\hat f(\theta) &=& - \ln \sum_{n=1}^{N}
\frac{\exp[-u(\x_{n};\theta)]}{\sum\limits_{k=1}^K N_k \, \exp[\hat{f}(\theta_{\rm ref,k}) - u(\x_{n};\theta_{\rm ref,k})]} \label{equation:estimator-of-free-energies}
\end{eqnarray}
where $K$ is the number of reference force fields, $N = \sum_k N_k$ is the total number of snapshots for all $K$ reference force fields, $N_k$ are the total number of snapshots from the $k^{th}$ reference force field, $\theta_{\rm ref,k}$ is the $k^{th}$ reference (i.e. simulated) force field, and $u(\x_{n};\theta) = \beta U(\x_n;\theta)$ is the reduced potential energy evaluated with $\theta$ for configuration $\x_n$  where $\beta = \frac{1}{k_{\rm B}T}$ and $k_{\rm B}$ is the Boltzmann constant. 

Note that $\hat f(\theta_{\rm ref,k})$ is required to evaluate the denominator of Equations \ref{MBAR weights theta}-\ref{MBAR free energy theta}. The values for $\hat f(\theta_{\rm ref,k})$ are obtained by solving a system of $K$ equations for self-consistency. Specifically, an initial guess for $\hat f(\theta_{\rm ref,k})$ is used to evaluate Equation \ref{MBAR free energy theta} with $\theta=\theta_{\rm ref,k}$ to obtain updated values of $\hat f(\theta_{\rm ref,k})$. This process is repeated until the values for $\hat f(\theta_{\rm ref,k})$ converge to within a desired tolerance. Although solving the MBAR system of equations for self-consistency may require several iterations, fortunately, once this process has been performed $\hat f(\theta)$ (for an arbitrary $\theta$) is evaluated without further iteration.

For the specific case of predicting $U^{\rm dep}$ and $Z$ for a non-sampled force field, expressed by the set of force field parameters $\theta$, the MBAR-based estimators for the departure internal energy and compressibility can be written as:
\begin{eqnarray} \label{MBAR U theta}
\langle U^{\rm dep}(\theta)\rangle = \sum_{n=1}^{N} U^{\rm dep}(\x_{n};\theta) W_{n}(\theta) 
\end{eqnarray}
\begin{eqnarray} \label{MBAR Z theta}
\langle Z(\theta)\rangle = \sum_{n=1}^{N} Z(\x_{n};\theta) W_{n}(\theta) 
\end{eqnarray}
where the energies and forces are computed using force field $\theta$ for each configuration $(\x_{n})$ to determine $U^{\rm dep}(\x_{n};\theta)$ and $Z(\x_{n};\theta)$ (from the virial pressure \cite{Allen1987}), respectively, while the weights $(W_n(\theta))$ are again calculated using Equations \ref{MBAR weights theta}-\ref{MBAR free energy theta}.
The performance of MBAR depends strongly on good phase space overlap, meaning that the configurations sampled by the reference force field(s) must represent a significant portion of the ``true'' configurations that the non-simulated force field would sample \cite{naden:jctc:2016}. If the configurational overlap is small, the MBAR estimates are often dominated by a few configurations, which are likely not representative of the ensemble that would be generated by direct simulation of force field $\theta$. The amount of overlap can be quantified by the number of effective samples $(N_{\rm eff})$ \cite{Dybeck2016}, using Kish's formula:
\begin{eqnarray} \label{Neff}
N_{\rm eff} = \frac{\left(\sum_n W_n\right)^2}{\sum_n W_n^2}
\end{eqnarray}
which reduces to $N_{\rm eff} = (\sum_n W_n^2)^{-1}$ when the weights are
normalized. This has the property that when the weights are equal,
$N_{\rm eff} = N$, when all but one weight is negligible, $N_{\rm eff} \approx
1$, and behaves appropriately for intermediate cases. 
In the case of poor overlap $(N_{\rm eff} \approx 1)$, the predicted values of MBAR will demonstrate a strong bias and the uncertainties will likely be underestimated by the MBAR covariance matrix. 

\section{Description of questions}

What degree of complexity is required to develop a ``technical quality'' force field? 

This research will address the following questions. Can a pairwise additive force field adequately represent the Helmholtz free energy and its derivatives over the entire fluid region of $P \rho T$ space? In other words, is there a non-bonded potential form that can match the accuracy of the REFPROP FEOS? If the extended Lennard-Jones is capable of meeting the task at hand, how many terms are required? Which terms provide the greatest improvement?  How many terms are needed  that can match the REFPROP FEOS 
The primary questions to b

Considering their complexity, it is unlikely that we will be able to resolve either of these questions during a two year postdoctoral fellowship. However, we will provide som 

For nearly half a century the Lennard-Jones 12-6 potential has inundated the molecular simulation literature. Only in the past decade has the Mie $n$-6 potential received considerable attention. By demonstrating significant improvement at high pressures, the aim of this research is to initiate a paradigm shift in non-bonded potentials, namely, the dawn of the extended Lennard-Jones era.

An FEOS correlation with ``technical accuracy'' is considered reliable over the entire fluid region of technological interest. The primary question of this study is whether or not it is possible to develop a pairwise additive force field of ``technical accuracy.'' Specifically, if enough terms are included in Equation BLANK (ex-LJ), can the force field fit higher order derivatives of the Helmholtz free energy over the entire fluid region of technological interest. Some intermediary questions that will be addressed are, which terms of Equation BLANK provide the greatest improvement in the force field? Without over-fitting, how many terms are required? 

As the ex-LJ has received relatively little attention in the literature, some additional questions are: which coefficients should be negative? To facilitate fitting and physical interpretation, should traditional parameters such as $\epsilon$, $\sigma$, etc. be used to describe the ex-LJ? What type of combining rules should be used when more than two terms are included in the expansion?

We identify four deliverables from this project. First, an improved FEOS for the molecules studies. Second, a force field of ``technical accuracy.'' Third, an increased theoretical understanding of non-bonded potentials. Fourth, an infrastructure for rapid force field parameterization using Helmholtz free energy derivatives. 

The potential impact of this study 

The  

 
FEOS correlations with ``technical acc

\end{document}
