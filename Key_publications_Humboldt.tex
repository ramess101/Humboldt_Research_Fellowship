\documentclass[12pt]{article}
% This is a list of the publications

\usepackage[letterpaper, portrait, margin=1in]{geometry}
\usepackage{cite}
\usepackage{amsfonts}
\usepackage{graphicx}
\usepackage{amsmath}
\usepackage{chemfig}
\usepackage{indentfirst}
\usepackage{setspace}
\usepackage{etaremune}

%\doublespacing

\title{List of the selected key publications of Richard Messerly}
\date{}

\renewcommand\thesection{\Alph{section}}

\newcommand\blfootnote[1]{%
	\begingroup
	\renewcommand\thefootnote{}\footnote{#1}%
	\addtocounter{footnote}{-1}%
	\endgroup
}

\begin{document}

\maketitle

\pagenumbering{gobble}

\begin{enumerate}
	
\item \textbf{Richard A. Messerly}, S. Mostafa Razavi, and Michael R. Shirts. ``Configuration-sampling-based surrogate models for rapid parameterization of non-bonded interactions.'' \textit{Journal of Chemical Theory and Computation.} 14 (6), 3144-3162, 2018.

This was my first publication during my National Research Council (NRC) postdoctoral associateship at the National Institute of Standards and Technology (NIST). Furthermore, this publication marked the first time that my co-authors were collaborators outside of my current institution. Furthermore, Michael R. Shirts is an extremely well-published young professor and S. Mostafa Razavi is a graduate student for J. Richard Elliott, an established expert in the field. Thus, this was a landmark publication that helped to proliferate my name throughout the molecular simulation community. With respect to the actual content, this publication compared Michael Shirt's existing method (MBAR) with a novel approach I developed (PCFR). The comparison demonstrated that these two methods are complementary and should be implemented together. Furthermore, I demonstrated how to merge these two methods with an approach developed by J. Richard Elliott (ITIC). \textbf{Personal contribution: 75\%.}

\item \textbf{Richard A. Messerly}, Thomas A. Knotts IV, and W. Vincent Wilding. ``Uncertainty quantification and propagation of errors of the Lennard-Jones 12-6 parameters for \textit{n}-alkanes.'' \textit{The Journal of Chemical Physics.} 146, 194110, 1-16, 2017.

This publication was essentially the culmination of my dissertation, as it led directly to my postdoctoral position at NIST. The main topic of this document, uncertainty quantification of molecular simulation predictions, is a strong interest of the Thermodynamics Research Center (TRC) at NIST. After presenting this research at a conference, the TRC group leader, Ken Kroenlein, strongly encouraged me to apply for an NRC postdoctoral position in his group. Following his recommendation was a significant step in developing my academic profile. Furthermore, this publication gathered the attention of the Open Force Field initiative, a group of renowned researchers that focuses on implementing Bayesian inference methods as part of force field development. As a participant in this initiative, I have created strong connections with a diverse set of experts in the field of molecular simulation. \textbf{Personal contribution: 90\%.}

\item \textbf{Richard A. Messerly}, Michael R. Shirts, and Andrei F. Kazakov. ``Uncertainty quantification confirms unreliable extrapolation toward high pressures for united-atom Mie $\lambda$-6 force field.'' \textit{The Journal of Chemical Physics.} (publisher's acknowledgment of receipt enclosed)

This study was motivated by the remarkable ``hybrid data set'' work performed by the host, Jadran Vrabec. More importantly, the results from this study were influential in my application to the Humboldt Research Fellowship. Specifically, this publication provided convincing evidence that the state-of-the-art Mie $\lambda$-6 force field is overly-repulsive at high pressures and, therefore, it should not be used with the hybrid data set approach. Naturally, this conclusion sparked several ideas for how to improve the force field. After discussing this issue with several colleagues, it was proposed that a more flexible (and theoretical) extended Lennard-Jones potential should perform better at high pressures. The use of the extended Lennard-Jones potential for hybrid data sets is the premise of my current proposal. \textbf{Personal contribution: 90\%.} 

%\item Edward J. Maginn, \textbf{Richard A. Messerly}$^*$, Daniel J. Carlson, Daniel R. Roe, J. Richard Elliott. ``Best Practices for Computing Transport Properties 1. Self-Diffusivity and Viscosity from Equilibrium Molecular Dynamics v1.'' \textit{Living Journal of Computational Molecular Science.} (publisher's acknowledgment of receipt enclosed)


\end{enumerate}

\end{document}