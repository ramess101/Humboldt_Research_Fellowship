\documentclass[11pt,a4paper]{article}
\usepackage{graphicx}
% uncomment according to your operating system:
% ------------------------------------------------
\usepackage[latin1]{inputenc}    %% european characters can be used (Windows, old Linux)
%\usepackage[utf8]{inputenc}     %% european characters can be used (Linux)
%\usepackage[applemac]{inputenc} %% european characters can be used (Mac OS)
% ------------------------------------------------
\usepackage{authblk}
\usepackage[superscript]{cite}
\usepackage[document]{ragged2e}
\usepackage[T1]{fontenc}   %% get hyphenation and accented letters right
\usepackage{mathptmx}      %% use fitting times fonts also in formulas
% do not change these lines:
\pagestyle{empty}                %% no page numbers!
\usepackage[left=35mm, right=35mm, top=15mm, bottom=20mm, noheadfoot]{geometry}
%% please don't change geometry settings!

\usepackage{fullpage}
\usepackage{amsfonts}
\usepackage{graphicx}
\usepackage{float}
\usepackage{amsmath}
\usepackage{chemfig}
\usepackage{indentfirst}
\usepackage{longtable}
\usepackage{array}
\usepackage{cellspace}
\usepackage{palatino}
%\usepackage{breqn}
\usepackage{amssymb}
\usepackage{verbatim}
\usepackage[colorlinks=true,citecolor=blue,linkcolor=blue]{hyperref}
\usepackage{siunitx}
\usepackage{xr}

% italicized boldface for math (e.g. vectors)
\newcommand{\bfv}[1]{{\mbox{\boldmath{$#1$}}}}
% non-italicized boldface for math (e.g. matrices)
\newcommand{\bfm}[1]{{\bf #1}}          

%\newcommand{\bfm}[1]{{\mbox{\boldmath{$#1$}}}}
%\newcommand{\bfm}[1]{{\bf #1}}
\newcommand{\expect}[1]{\left \langle #1 \right \rangle} % <.> for denoting expectations over realizations of an experiment or thermal averages

\newcommand{\var}[1]{{\mathrm var}{(#1)}}
\newcommand{\x}{\bfv{x}}
\newcommand{\y}{\bfv{y}}
\newcommand{\f}{\bfv{f}}

\newcommand{\hatf}{\hat{f}}

\newcommand{\bTheta}{\bfm{\Theta}}
\newcommand{\btheta}{\bfm{\theta}}
\newcommand{\bhatf}{\bfm{\hat{f}}}
\newcommand{\Cov}[1] {\mathrm{cov}\left( #1 \right)}
\newcommand{\T}{\mathrm{T}}                                % T used in matrix transpose

\newcommand\blfootnote[1]{%
	\begingroup
	\renewcommand\thefootnote{}\footnote{#1}%
	\addtocounter{footnote}{-1}%
	\endgroup
}

\bibliographystyle{unsrt}

% begin the document
\begin{document}
	\thispagestyle{empty}
	%make title bold and 14 pt font (Latex default is non-bold, 16 pt)
	\title{\Large \textbf{Humboldt Research Fellowship}}
	\author[1]{\large {\underline{Richard Messerly}}}%%[12 pt regular, presenting speaker underlined]
	
	\affil[1]{\textit{Thermodynamics Research Center (TRC), National Institute of Standards and Technology (NIST),
			Boulder, Colorado, 80305, USA}}
	
	\date{} % <--- leave date empty
	\maketitle\thispagestyle{empty} %% <-- you need this for the first page
%	\begin{center}
%		\title{\textbf{ABSTRACT}}\centering{}
%	\end{center}
	\justify

\section*{Description of proposal guidelines on HRF website}

The current state of research should first be briefly described and supported by approximately five revelant publications from the research area (one page max).

The outline should focus on a clear description of the questions you intend to address in your research, their originality and significance for the advancement of the research field (approx. two pages).

Furthermore, the academic methods to be used to achieve these goals should be clearly described and referenced, if appropriate (approx. two pages).

The research outline should comprise approximately five pages in total (including references). Should you significantly exceed this length, you may be asked to cut it down to approximately five pages.

For the purposes of evaluation it must be clearly demonstrated that you yourself have drawn up the main contents independently and agreed them beforehand with your host. Any contents contributed by the host institute must be attributed accordingly.

\section*{Rough draft: Self-consistent fundamental equations of state and force fields of ``technical accuracy''}

The design of efficient and reliable chemical processes requires accurate prediction of thermophysical properties over a wide range of temperatures and pressures. Fundamental equations of state (FEOS), such as those based on the Helmholtz free energy ($A$), are a powerful approach for estimating pressure, density, temperature $(P\rho T)$ behavior and caloric properties, such as internal energy $(U)$ and isochoric/isobaric heat capacities $(c_{\rm v}$ and $c_{\rm p}$, respectively). In general, the Helmholtz free energy is separated into an ideal gas $(A^{\rm ig})$ and a residual $(A^{\rm r})$ contribution. As $\alpha^{\rm ig}$ is typically obtained using first principles (\textit{ab initio}) calculations, the primary focus of FEOS development is modeling $A^{\rm r}$. State-of-the-art FEOS based on Helmholtz free energy utilize a semi-empirical model for $A^{\rm r}$ with between 50 and 100 non-linear fitting parameters. 

An FEOS correlation is considered to be of ``technical accuracy'' when it is reliable over the entire fluid region of technological interest. Unfortunately, most compounds do not have sufficient \textit{reliable} experimental data for a diverse set of thermodynamic properties covering a wide range of $P \rho T$ conditions to develop a technical accuracy FEOS. Mixture FEOS often require additional fitting parameters although experimental data are even more scarce and of questionable quality. Due to the amount of fitting parameters, FEOS predictions can result in large errors when extrapolated to higher temperatures and pressures than those of the training data. Improvement in an FEOS at high temperatures and pressures necessitates additional data for those conditions. The lack of experimental data at high temperatures and pressures, especially, is attributed to the inherent safety, cost, and complexity of such experiments.

By contrast, molecular simulation (i.e. Monte Carlo, MC, and molecular dynamics, MD) methods at high temperatures and pressures do not suffer from any of these limitations. For this reason, the host and collaborators have utilized ``hybrid data sets'' consisting of both available experimental data and molecular simulation results at extreme temperatures and pressures to extend the FEOS range of applicability for several compounds \cite{Thol2016_siloxane_first,Thol2016_siloxane,Thol2017,Rutkai2013,Thol2015}. The simulation values that are included in hybrid data sets are derivatives of the residual Helmholtz free energy with respect to inverse temperature and/or density :
\begin{equation} \label{Residual Derivatives}
A_{xy}^{\rm r} R_gT \equiv (1/T)^x\rho^y \frac{\partial^{x+y}A^{\rm r}}{\partial(1/T)^x \partial \rho^y}
\end{equation}
We would like to emphasize the advantage of using $A_{xy}^{\rm dep}$ for developing FEOS, as this choice eliminates redundant information found in traditional macroscopic properties \cite{Thol2016_LJ,Thol_LJTS,Rutkai2017,Lustig2015,Rutkai2013,Rutkai2015}. Furthermore, experimental data are typically measured for properties that relate only to first and second derivatives, whereas, in principle, molecular simulation provides an avenue for estimating higher order derivatives, which are invaluable when fitting FEOS. Note that \textit{ms2} \cite{ms2}, the open-source simulation package developed by Jadran Vrabec's group, already computes $A_{xy}^{\rm r}$. Therefore, the infrastructure is already in place to implement the hybrid data set approach at the Technische Universitat Berlin.
 
%The inclusion of $A_{xy}^{\rm r}$ simulation results greatly improved the extrapolation of the FEOS to extreme temperatures and pressures.
%Recently,  successfully applied this ``hybrid data set'' approach to dramatically extend the FEOS' range of applicability for several compounds \cite{Rutkai2013,Thol2016_siloxane_first,Thol2016_siloxane,Thol2017,Thol2015}.

The primary limitation of this approach is that most force fields are not ``transferable'' over a wide range of $P \rho T$ conditions. For example, in a previous study, we demonstrated the poor transferability to high pressures of the popular united-atom Mie $n$-6 potential (of which the traditional Lennard-Jones 12-6 is a subclass). This deficiency in the force field causes the FEOS to be inconsistent with the force field. The primary question that this study will address, at least in part, is whether or not it is possible to develop a pairwise additive force field of the same ``technical accuracy'' as the FEOS. As thermodynamic properties are extremely sensitive to the intermolecular interactions, we propose that more theoretical and flexible non-bonded potential functions be considered when developing FEOS. For example, two body potentials developed from \textit{ab initio} values typically utilize more than ten fitting parameters because of the high information content in \textit{ab initio} calculations at different intermolecular distances (see Equation 7 of Reference \cite{Przybytek2017}).

% we focus primarily on developing extremely flexible intermolecular potential functions.

%The poor extrapolation was due to $n > 12$ leading to overly repulsive forces at high densities.

% this is that the optimal value of $n$ to reproduce saturated vapor and liquid properties is typically greater than 12 (e.g. the Potoff 16-6 potential). Unfortunately, $n \gg 12$ is not theoretically justified and leads to an overly repulsive potential at high densities.

%To improve the performance of this so-called ``hybrid data set'' approach, it is imperative that the force field be transferable over different $P \rho T$ conditions. Therefore, we propose that more theoretical and flexible potential models be considered for developing FEOS. Specifically, we propose the development of extended Lennard-Jones (ex-LJ) force fields to improve the fundamental equations of state at high pressures.

%In a previous study, we demonstrated the poor transferability to high pressures of the popular united-atom Mie $n$-6 potential (of which the traditional Lennard-Jones 12-6 is a subclass). 

%To improve this method, we propose that more theoretical and flexible non-bonded potential functions be considered for developing FEOS. For example, two body potentials developed from \textit{ab initio} values typically utilize five to ten fitting parameters because of the high information content in \textit{ab initio} calculations at different intermolecular distances.

In this study, we propose the development of extended Lennard-Jones (ex-LJ) force fields to improve performance at high pressures. The most general expression for the extended Lennard-Jones non-bonded potential is:
\begin{equation} \label{ex-LJ general}
u_{\rm nb, exLJ} = \sum_m C_m r^{-m} 
\end{equation} 
where $m$ are integer values and $C_m$ are the coefficients for the corresponding $r^{-m}$ terms. Note that the traditional Lennard-Jones 12-6 potential is obtained if $C_{12} = 4\epsilon\sigma^{12}$ and $C_6=4\epsilon\sigma^{6}$, while all other $C_m$ values are zero. 

For nearly half a century the Lennard-Jones 12-6 potential has inundated the molecular simulation literature. Only in the past decade has the Mie $n$-6 potential received considerable attention. Note that the popular choice of $n =12 $ for the Lennard-Jones 12-6 potential is a historical artifact based primarily on computational reasons that are no longer significant. The ex-LJ potential is more flexible than the two-parameter ($\epsilon$ and $\sigma$) LJ 12-6, and more theoretically justified than the three-parameter Mie $n$-6, particularly when $n \gg 12$. However, it has not been tested as extensively as the LJ 12-6, Mie $n$-6, and exponential-6 potentials. By demonstrating significant improvement at high pressures, the impact of this research would be to initiate a paradigm shift in non-bonded potentials. The question remains, when enough non-zero terms are included in Equation \ref{ex-LJ general}, can an ex-LJ force field fit higher order derivatives of the Helmholtz free energy over the entire fluid region of technological interest? The development of ``technical accuracy'' ex-LJ force fields allows for prediction of other properties not available from the FEOS, such as transport properties (e.g., self-diffusivity, shear viscosity) and structural/kinetic phenomenon, as well as various mixture properties. 

%Traditionally, there are only two to four non-zero $C_m$ values, e.g., 16-6, 12-8-6, 14-12-8-6. The primary reasons for this are to simplify the parameterization and to avoid over-fitting if the data are insufficient to characterize more parameters. In principle, however, any number of $C_m$ terms could be non-zero. 

As the ex-LJ has received relatively little attention in the literature, some additional questions naturally arise. For example, which coefficients should be negative? Which terms of Equation \ref{ex-LJ general} provide the greatest improvement in the force field? How many terms should be non-zero such that the model is sufficiently flexible but not over-fit? What type of combining rules should be used for cross interactions? 

The $r^{-6}$, $r^{-8}$, and $r^{-10}$ terms can be derived from dispersive interactions and, therefore, it seems most reasonable for $C_6$, $C_8$, and $C_{10}$ to be negative. However, it is unclear how well a 12-10-8-6 potential would perform if only the $C_{12}$ coefficient is positive. Therefore, from an empirical standpoint, it is possible to allow all coefficients to be either positive or negative.

%Without over-fitting, how many terms are required?

Although the ex-LJ was proposed over two decades previously, the main reason for the lack of popularity is the additional complexity in parameterizing the ex-LJ potential. Due to the additional model parameters of the ex-LJ, the traditional approach of performing molecular simulation with each proposed parameter set $(\theta)$ is computationally infeasible when several $C_m$ terms are non-zero. This presents another essential question, what optimization method is best suited for parameterizing the ex-LJ potential?

In previous hybrid data set studies, the non-bonded parameters were optimized with VLE properties. However, it is unlikely that VLE data alone can provide a unique set of parameters for more than three non-zero $C_m$ terms. By contrast, derivatives of Helmholtz free energy ($A_{xy}^{\rm r}$) provide high information content regarding the non-bonded potential. Unfortunately, reliable $A_{xy}^{\rm r}$ values require an accurate FEOS. For this reason, we propose an iterative hybrid data set approach where the number of non-zero $C_m$ terms increases at each iteration:

\begin{enumerate}
	\item Develop a FEOS over the temperatures and pressures where reliable experimental data exist
	\item Parameterize the ex-LJ force field with $A_{xy}^{\rm r}$ in range where FEOS is reliable
	\item Iterate:
	\begin{enumerate}
		\item Estimate $A_{xy}^{\rm r}$ for ex-LJ force field at state points where experimental data are not available
		\item Refit the FEOS to the hybrid data set
		\item Re-parameterize the ex-LJ force field using updated FEOS $A_{xy}^{\rm r}$
	\end{enumerate}
\end{enumerate}

% It remains to be seen how many state points  unique non-zero $C_m$ terms can be parameterized simultaneously.


%Basis functions are amenable to the extended Lennard-Jones potential. For example, the 

%Basis functions store the contributions from the $r^{-m}$ terms for the nonbonded energies and forces

%In previous hybrid data set studies, the LJ 12-6 parameters were optimized with VLE properties. The two reasons for this practice are the availability of VLE data and the difficulty of optimizing force field parameters. If several model parameters are included in Equation BLANK, the traditional approach of force field parameterization, namely, trial-and-error optimization is not computationally feasible. Furthermore, it is unlikely that VLE data alone can provide a unique set of parameters. By contrast, derivatives of Helmholtz free energy provide high information content regarding the non-bonded potential. 

%The methodology proposed in this study is to:
%\begin{enumerate}
%	\item Develop a FEOS over the temperatures and pressures where reliable experimental data exist
%	\item Fit the ex-LJ parameters to FEOS derivative Helmholtz free energy properties in range where FEOS is reliable
%	\item Perform molecular simulations at state points where experimental data are not available
%	\item Iterate:
%	\begin{enumerate}
%		\item Refit the FEOS to the hybrid data set
%		\item Re-optimize the force field parameters to match FEOS (using MBAR so that no additional molecular simulations are required)
%	\end{enumerate}
%\end{enumerate}

With this iterative approach, it is possible to ensure that the Helmholtz free energy derivative properties are internally consistent between the FEOS and the ex-LJ force field. The purpose for this iterative approach is to improve the extrapolation of FEOS and the reliability of the force field. As the FEOS and ex-LJ force field become more self-consistent with each successive iteration, it is possible to increase the number of non-zero $C_m$ terms in Step 3c, although this may necessitate including higher order derivatives of $A_{xy}^{\rm r}$.

%fitting parameters in both the FEOS and in the ex-LJ. 

%The iterative hybrid data set approach utilizes allows for non-bonded potential parameters to be fit with high order derivative terms from Equation BLANK can be used to provide non-redundant information for fitting the non-bonded potential parameters. 
Step 3c of this algorithm is the computational bottleneck when direct molecular simulations are performed for each re-parameterization of the ex-LJ force field. In fact, the traditional brute-force trial-and-error optimization approach is not computationally feasible for more than three non-zero $C_m$ terms. To facilitate parameterization of ex-LJ potentials with $A_{xy}^{\rm r}$, we propose the use of Multistate Bennett Acceptance Ratio (MBAR) combined with basis functions $(\Phi)$. Previous publications demonstrate that MBAR-$\Phi$ reduces the computational cost to estimate ensemble averages by several orders of magnitude compared to direct molecular simulation. Therefore, utilizing MBAR-$\Phi$ in Steps 3c is essential for this algorithm to be computationally tractable. In addition, MBAR-$\Phi$ can be applied in Step 3a to eliminate the need to re-simulate the additional state points for each iteration. 

%The combined MBAR-$\Phi$ approach estimates ensemble averages for MBAR has been shown to be a reliable approach for reweighting conf has been shown to greatly accelerate force field parameterization

MBAR is a statistical method that reweights configurations sampled with a reference force field(s) to predict ensemble averages for a non-simulated force field \cite{shirts-chodera:jcp:2008:mbar,Messerly2018_1}. For example, derivatives of the Helmholtz free energy for parameter set $\theta$ are estimated according to 
\begin{equation}
A_{xy}^{\rm r}(\theta) = F_i[\langle U(q_{\rm ref},\theta)^i \rangle_{\rm MBAR}] + F_j[\langle U(q_{\rm ref},\theta)^j \rangle_{\rm MBAR}]
\end{equation}
where $q_{\rm ref}$ are configurations sampled using the reference force field(s), $\langle\rangle_{\rm MBAR}$ are ensemble averages estimated using MBAR (see Equations 9 to 11 of Reference \cite{Messerly2018_1}), and $F_i$ and $F_j$ are functionals that depend on different powers of the internal energy (see Equations 27 and 30 of Reference \cite{Lustig2012}). 

While MBAR reweighting is orders of magnitude faster than performing a direct molecular simulation, MBAR requires ``recalculating'' the non-bonded energies for each configuration sampled. Basis functions greatly accelerate the cost of this recalculation step by several orders of magnitude compared to the Gromacs ``rerun'' function. The total non-bonded internal energy $(U_{\rm nb, total})$ is computed from $\sum_{i=1}^{N_{\rm sites}-1} \sum_{j=i+1}^{N_{\rm sites}} \sum_m C_{m,ij} r_{ij}^{-m}$, where $N_{\rm sites}$ is the number of interacting sites in the system, $C_{m,ij}$ is the $C_{m}$ term for the $ij$ interaction and $r_{ij}$ is the intermolecular distance between sites $i$ and $j$. Due to the linear relationship between $U_{\rm nb, total}$ and Equation \ref{ex-LJ general}, the ex-LJ potential is amenable to basis functions. Rather than storing the configurations of all $N$ molecules, basis functions store the $\sum_{i=1}^{N_{\rm sites}-1} \sum_{j=i+1}^{N_{\rm sites}} r_{ij}^{-m}$ contributions for different values of $m$. Therefore, recomputing $U_{\rm nb, total}$ for a different set of $C_{m,ij}$ parameters $(\theta)$ can be performed with simple and \textit{fast} matrix multiplication.

Four deliverables are expected from this project. First, a ``technical accuracy'' FEOS for each molecule studied. Second, an extended Lennard-Jones force field of ``technical accuracy.'' Third, an increased theoretical understanding of non-bonded potentials. Fourth, an infrastructure for rapid force field parameterization using Helmholtz free energy derivatives. 

\bibliography{Humboldt_references}

\end{document}
