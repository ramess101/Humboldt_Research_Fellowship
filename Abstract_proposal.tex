\documentclass[12pt]{article}
% This is a list of the publications

\usepackage[letterpaper, portrait, margin=1in]{geometry}
\usepackage{cite}
\usepackage{amsfonts}
\usepackage{graphicx}
\usepackage{amsmath}
\usepackage{chemfig}
\usepackage{indentfirst}
\usepackage{setspace}
\usepackage{etaremune}

%\doublespacing

\title{Short abstract}
\date{}

\renewcommand\thesection{\Alph{section}}

\newcommand\blfootnote[1]{%
	\begingroup
	\renewcommand\thefootnote{}\footnote{#1}%
	\addtocounter{footnote}{-1}%
	\endgroup
}

\begin{document}
	
%	\maketitle
	
	\pagenumbering{gobble}
	
	\section*{Short abstract}
	
	\subsection*{Online description}
	
	The short abstract of your research proposal should be worded in a manner that is appropriate for non-specialist academics and should summarize the main goals and contents of the research you plan to do in Germany. Please do not use abbreviations without explaining them.
	
	\subsection*{My text}
	
	Reliable estimates of thermophysical properties are essential for designing efficient and reliable technical processes. Fundamental equations of state (FEOS) based on the Helmholtz energy allow for prediction of pressure, density, temperature behavior as well as energetic properties, e.g., heat capacities. Unfortunately, most molecular species (and, to a greater extent, mixtures) do not have enough reliable experimental data to fit the large number of FEOS parameters. In this case, molecular simulation can supplement experimental data at state points where reliable data are scarce, typically at high temperatures and pressures. 
	
	The primary limitation of this so-called ``hybrid data set'' approach is the accuracy of the force field used in the molecular simulation. Specifically, most force fields perform well for vapor-liquid equilibrium properties but extrapolate poorly to high pressures. As thermophysical properties are highly sensitive to the non-bonded interactions, we propose using the extended Lennard-Jones (ex-LJ) potential, which is significantly more flexible than the traditional Lennard-Jones 12-6 potential. Furthermore, we propose an iterative hybrid data set approach, where the ex-LJ parameters are re-optimized after each iteration to ensure self-consistency between the FEOS and the force field.
	
	To reduce the computational cost of this iterative approach, we will implement Multistate Bennett Acceptance Ratio (MBAR) combined with basis functions. In my previous work, I demonstrated that MBAR with basis functions yields extremely fast and reliable estimates of thermophysical property values for any force field parameter set, without performing direct molecular simulation. My expertise with MBAR and basis functions, the host's simulation infrastructure and methods, and our close collaborations with experts in FEOS development are essential for the success of this project.
	
	\section*{What do you think will be the impact of your research on the further development of your academic profile?}
	
	I intend on making the proposed methodology, ``iterative hybrid data sets,'' the keystone of my future research. It would be impossible to develop equations of state and force fields for every compound of interest during this two-year fellowship. Therefore, I will continue to implement this approach for years to come with additional molecular species.
	
	My long-term career path is to become a professor, but with a strong emphasis on ``industrially relevant'' research. The proposed research is exemplary of this as it utilizes state-of-the-art scientific/academic methods but with the ultimate impact found in industry. The primary benefit of this research is that it allows me to pursue a career path in academia, industry, or a government agency.
	
	Working with Vrabec, a pioneer in hybrid data sets, will greatly accelerate and deepen my understanding of this approach. Furthermore, I will explore the details and better appreciate the challenges of fitting equations of state with high-dimensional non-linear models. This skill will be invaluable throughout my career, regardless of the path I pursue. I will learn Fortran 90, an extremely valuable coding language, as I contribute to the molecular simulation package developed by Vrabec's group, \textit{ms}2. By working directly with the developers of \textit{ms}2, I also expect that my coding practices will improve.
	
	As my previous research groups were relatively small, joining Vrabec's group is a great opportunity for me to see firsthand how larger research groups function. Under Vrabec's tutelage, I will develop my own style for teaching, mentoring, managing several projects, and performing research. Being at a German university fosters both a diverse environment and new global connections.
		
	In brief, there are four key facets in which the proposed research will significantly impact the development of my academic profile:
	
	\begin{enumerate}
		\item Expertise/skills
		\item Leadership
		\item Diversification
		\item Networking
	\end{enumerate}
		
	%Mentoring undergraduates and doctoral candidates will improve my leadership and teaching skills.	
		
	% collaborate closely with industry, government research agencies, and academia. Thus, following my fellowship, I am able to pursue whichever career path I desire.

	
		% can either pursue a career path in industry, continue in academia, or return to NIST. 
	
	%, and I will contribute to \textit{ms}2, the molecular simulation package developed by Vrabec's group.
	  
%	for  As I receive tutelage   The host institute is the ideal location for this research since Vrabec's group developed the hybrid data set approach,   
	
%	Although I have collaborated with the NIST group that develops equations of state (REFPROP), I have never developed one myself. This research will allow me to explore the details and better understand the challenges of fitting these high-dimensional non-linear models. 
	
	%This skill would be invaluable if, following my fellowship, I return to academia or NIST or if I pursue a career in industry.
	 
%	, and  Despite my familiarity with various simulation codes and several coding languages, this presents a new challenge/opportunity for me.

  

%	
%	Diversification/Networking:
%	
%	\begin{enumerate}
%		\item My first postdoctoral position was at a US government agency (NIST). I will create new connections with a postdoctoral position at a German university.
%		\item I earned my doctorate at the same university where I completed my bachelors degree. This will help me develop my own methods for teaching and performing research. 
%	\end{enumerate}

%My long-term career path is to become a professor, although I continue to have a strong emphasis on "industrially relevant" research. The proposed research is exemplary of applying state-of-the-art scientific/academic methods but with the ultimate impact found in industry. The primary benefit from this research domain is that it allows me to collaborate closely with industry, government research agencies, and academia. In particular, there are several facets in which the proposed research will significantly impact the development of my academic profile:
%
%Skills development:
%
%\begin{enumerate}
%	\item Although I have collaborated with the NIST group that develops equations of state (REFPROP), I have never developed one myself. This research will allow me to explore the details and better understand the challenges of fitting these high-dimensional non-linear models. This skill would be invaluable if, following my fellowship, I return to academia or NIST or if I pursue a career in industry.
%	\item I will learn Fortran 90, an extremely valuable coding language, and ms2, the molecular simulation package developed by Vrabec's group. Despite my familiarity with various simulation codes and several coding languages, this presents a new challenge/opportunity for me.
%\end{enumerate}
%
%Leadership development:
%
%\begin{enumerate}
%	\item I have only mentored undergraduate students. Mentoring doctoral candidates will improve my leadership skills.
%	\item My previous groups were small compared to Vrabec's. Joining Vrabec's group is a great opportunity for me to see firsthand how a larger research group functions.  
%\end{enumerate}
%
%Diversification/Networking:
%
%\begin{enumerate}
%	\item My first postdoctoral position was at a US government agency (NIST). I will create new connections with a postdoctoral position at a German university.
%	\item I earned my doctorate at the same university where I completed my bachelors degree. This will help me develop my own methods for teaching and performing research. 
%\end{enumerate}

	
\end{document}