\documentclass[12pt]{article}

\usepackage[letterpaper, portrait, margin=1in]{geometry}
\usepackage[numbers,sort&compress,square,super]{natbib}
\usepackage{amsfonts}
\usepackage{graphicx}
\usepackage{amsmath}
\usepackage{chemfig}
\usepackage{indentfirst}
\usepackage{setspace}

\doublespacing

\begin{document}

The design of an efficient chemical process requires accurate thermophysical property data with corresponding uncertainties. For this reason, the National Institute of Standards and Technology (NIST) Thermodynamics Research Center (TRC) provides evaluated thermodynamic data and correlations. With advances in computer technology in recent years, molecular simulation is an essential tool in this regard. For example, molecular simulation results can supplement and help discern between conflicting sets of experimental data. However, the accuracy in molecular simulation results depends upon the molecular model, i.e. the force field.

Although several force fields exist in the literature, most of these models were developed using trial and error parameterization to a small set of experimental data. Furthermore, the parameters for different functional groups are optimized in a sequential approach that can lead to poor extrapolation when applied to compounds and conditions not included in the training set. Also, these studies rarely report an assessment of uncertainties in the optimal parameters. Each of these deficiencies is intimately related to the high computational cost of molecular simulation. We propose a novel ``post-simulation optimization'' approach to overcome these shortcomings. Our goal is to utilize this method with the extensive, evaluated, experimental data archives available at NIST/TRC to develop an inclusive library of self-consistent, predictive force field models.


\end{document}
